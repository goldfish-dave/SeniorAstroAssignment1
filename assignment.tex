\documentclass[10pt,a4paper]{article}
\usepackage{fullpage}

\title{Senior Astrophysics 2011 Assignment 1}
\date{}
\author{D. G. Wilcox \\
		309248035}

\begin{document}
\maketitle
\section*{Question 1}

\subsection*{Ptolemy's Epicycles}
Ptolemy's epicycles was a geometric model used to explain the motion of the celestial bodies. It set out to explain the retrograde motion of the five planets known at the time, and the changes in the apparent distances of the planets from Earth.

Due to the belief at the time that the Earth was at the centre of the Universe the motion of the planets could not be described as simply as they can be if you take the Sun to be the centre of the solar system.

The motion of the planets is explained by assuming they orbit in epicycles which themselves movein circles. Unfortunately, in order for the model to accurately portray the motion of the planets "circles upon circles" had to be added.

\subsection*{Multiverse Theory}
When refering to 'Multiverse theory' outside of the physics lab it is important to clarify \textit{which} theory you are refering to. Because of the romantic appeal it posseses and its inherently abstractness is it a common target for science-fiction, pseudo-science, mysticism and philosophy. Additionally, even in the realm of physics many of the multiverse theories lack readily falsifiable predictions.

In short, a multiverse theory claims the possible existence of more than the current universe. These alternate universes often represent a universe like ours in which some property has changed or the result of some random outcome was different. Such a theory seeks to explain why certain observable characteristics of our universe are the way they are. If an electron had an equal probability to go through both slits, why did it go through slit B? If the gravitational constant (or any other 'constant' of physics) is not the result of a solvable equation, why is it what it is? A multiverse theory explains this by saying all possible outcomes are represented by all possible universes.

\subsection*{Comparison of Adequateness}
\begin{quote}
	"A theory is a good theory if it satisfies two requirements: It must accurately describe a large class of observations on the basis of a model that contains only a few arbitrary elements, and it must make definite predictions about the results of future observations."

	-- Stephen Hawking
\end{quote}

\section*{Question 2}
\section*{Question 3}
\section*{Question 4}
\end{document}
