\documentclass[10pt,a4paper]{article}
\usepackage{fullpage}

\title{Senior Astrophysics 2011 Assignment 1}
\date{}
\author{D. G. Wilcox \\
		309248035}

\begin{document}
\maketitle
\section*{Question 1}

\subsection*{Ptolemy's Epicycles}
Ptolemy's epicycles was a geometric model used to explain the motion of the celestial bodies. It set out to explain the retrograde motion of the five planets known at the time, and the changes in the apparent distances of the planets from Earth.

Due to the belief at the time that the Earth was at the centre of the Universe the motion of the planets could not be described as simply as they can be if you take the Sun to be the centre of the solar system.

The motion of the planets is explained by assuming they orbit in epicycles which themselves movein circles. Unfortunately, in order for the model to accurately portray the motion of the planets "circles upon circles" had to be added.

\subsection*{Multiverse Theory}
When refering to 'Multiverse theory' outside of the physics lab it is important to clarify \textit{which} theory you are refering to. Because of the romantic appeal it posseses and its inherently abstractness is it a common target for science-fiction, pseudo-science, mysticism and philosophy. Additionally, even in the realm of physics many of the multiverse theories lack readily falsifiable predictions.

In short, a multiverse theory claims the possible existence of more than the current universe. These alternate universes often represent a universe like ours in which some property has changed or the result of some random outcome was different. Such a theory seeks to explain why certain observable characteristics of our universe are the way they are. If an electron had an equal probability to go through both slits, why did it go through slit B? If the gravitational constant (or any other 'constant' of physics) is not the result of a solvable equation, why is it what it is? A multiverse theory explains this by saying all possible outcomes are represented by all possible universes.

\subsection*{Comparison of Adequateness}
In order to objectively compare these two theories we need an objective standard to measure them against. For this I have chosen a quote from Stephen Hawking:

\begin{quote}
	"A theory is a good theory if it satisfies two requirements: It must accurately describe a large class of observations on the basis of a model that contains only a few arbitrary elements, and it must make definite predictions about the results of future observations."

	-- Stephen Hawking
\end{quote}

This standard consists of two points: does the theory make falsifiable predictions, and does the model avoid excessive use of arbitrary assumptions. I will analyse both theories using both criteria.

Ptolemy's theory makes predictions about the motion of planets that can be proven to be either true or false by directly observing them (although the technology to do so at the time was possibly lacking). Even if Ptolemy modified his theory by adding the so called "circles upon circles" to increase his accuracy, if you model each of these modifications to his theory as a new theory then each of these theories are themselves falsifiable.

Where Ptolemy's theory falls short is its excessive use of arbitray claims. The claim to a geocentric model is assumed without evidence, as is the claim to perfectly circular orbits.

Multiverse Theory lacks falsifiable predictions by its very nature. So long as our universe is defined as "the totality of all things that exist" (or some similar variant), anything outside of our universe is unobservable through direct and indirect means. So until a multiverse theory provides evidence to the contrary it will only ever be a hypothesis.

Multiverse Theory also does poorly on the second front. We have as much reason to believe this is the only universe and random variables like the speed of light and the gravitational constant are truely random as we do to believe all possible outcomes exist.

\section*{Question 2}

\begin{itemize}
	\item[(a)] The first Friedmann equation:
	\begin{equation}
		(\frac{\dot a}{a})^{2} = \frac{8 \pi G \rho}{3} - \frac{kc^{2}}{a^{2}} \\
	\end{equation}
	with no cosmological constant. Rearranging to give in terms of $H(a)$, $H_{0}$, $a$ and $\Omega_{a}$:
\end{itemize}
\begin{eqnarray*}
	H^{2} =& \frac{8 \pi G \rho}{3} - \frac{kc^{2}}{a^{2}}, &H = \frac{\dot a}{a} \\
	  =& \frac{H_{0}^{2} \rho}{\rho_{crit}} - \frac{kc^{2}}{a^{2}}, &\frac{\rho_{crit}}{H_{0}^{2}} = \frac{3}{8 \pi G} \\
	  =& H_{0}^{2}\Omega - \frac{kc^{2}}{a^{2}}, &\Omega = \frac{\rho}{\rho_{crit}} \\
	  =& H_{0}^{2}[\frac{kc^{2}}{a^{2}H^{2}} + 1] - \frac{kc^{2}}{a^{2}}, & \Omega = \frac{kc^{2}}{a^{2}H^{2}} + 1 \\
	\Rightarrow (\frac{H}{H_{0}})^{2} =& \frac{kc^{2}}{a^{2}H^{2}} + 1 - \frac{kc^{2}}{a^{2}H_{0}^{2}}, &\mbox{divide by $H_{0}^{2}$}
\end{eqnarray*}
\begin{itemize}
	\item[(b)] 
	\item[(c)] Solve for $d/d\theta [a(\theta)] = 0$ to get $\theta = \pi$. Sub into $t(\theta)$:
\end{itemize}
\begin{eqnarray*}
	t(\pi) &=& \frac{1}{2}\frac{\Omega_{0}}{\Omega_{0} - 1} \frac{1}{H_{0}(\Omega_{0} - 1)^{0.5}} (\pi - 0) \\
	       &=& \frac{\pi}{2}\frac{\Omega_{0}}{H_{0}(\Omega_{0} - 1)^{1.5}} \\
		   &&\mbox{where $\Omega_{0} = 2$, $H_{0}^{-1} = 14 \times 10^{9}$ years} \\
		   &=& 14\pi \times 10^{9} \\
		   &\approx& 4.4 \times 10^{10} \mbox{years} \\
		   &&\mbox{and for when the Big Crunch occurs:} \\
	t(2\pi) &=& \frac{2\pi}{H_{0}} \\
		  &\approx& 8.8 \times 10^{10} \mbox{years}
\end{eqnarray*}
\begin{itemize}
	\item[(d)] Using:
\end{itemize}
\begin{eqnarray*}
	\frac{T}{T_{0}} &=& \frac{a}{a_{0}} \mbox{ at $T_{0} = 3$ K} \\
	T &=& \frac{a_{0}T}{a(\pi)} \\
	  &=& 1 \times 3 \times [\frac{1}{2}\frac{\Omega_{0}}{\Omega_{0} -1} (1 - \cos \theta)]^{-1} \\
	  &=& \frac{6}{2}(1 - \cos \pi)^{-1} \\
	  &=& 3/2 \mbox{ K}\\
	  \mbox{and when the Universe has shrunk back to its present size:} \\
	  T &=& 3 \mbox{ K}
\end{eqnarray*}

\section*{Question 3}
\begin{eqnarray*}
	q_{0} =& -\frac{1}{H_{0}^{2}}[\frac{-4\pi G \rho_{0}}{3} + \frac{\Lambda c^{2}}{3}], &\frac{\ddot a}{a} = \frac{-4\pi G \rho}{3} + \frac{\Lambda c^{2}}{3} \\
	=& -\frac{1}{H_{0}^{2}}[\frac{-1}{2}\Omega_{M} + H_{0}^{2}\Omega_{\Lambda}], &\Omega_{\Lambda} = \frac{\Lambda}{3H_{0}^{2}} \\
	=& \frac{1}{2}\frac{\Omega_{M}}{H_{0}^{2}} + \Omega_{\Lambda} &
\end{eqnarray*}
\section*{Question 4}
\end{document}
